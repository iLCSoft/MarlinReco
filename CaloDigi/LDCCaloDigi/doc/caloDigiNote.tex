\documentclass[12pt]{article} %[epsf]
\usepackage{amsmath}

\sloppy

\title{Realistic calorimeter hit digitisation in the {\tt ILDCaloDigi} processor}

\author{Daniel Jeans, \\ 
Dept of Physics, \\
Grad School of Science, \\
The University of Tokyo.}

\date{\today}

\begin{document}

\maketitle

\begin{abstract}

Description of quasi-realistic calorimeter hit digitisation in ILD, as implemented in
the {\tt ILDCaloDigi} processor in the {\tt MarlinReco} package.

\end{abstract}


\section{Introduction}

Possibilities for more realistic treatment of calorimeter hits from silicon-- and scintillator--based calorimeters have been 
implemented in the {\tt ILDCaloDigi} processor within {\tt MarlinReco}. The aim of these is to allow the study of the effects 
of various detector ``defects'' such as mis-calibrations, limited dynamic ranges, and signal fluctuations, 
and also to allow more robust comparisons between technologies under more realistic conditions.
This notes discusses the implementation as defined by rev.4744 of the {\tt MarlinReco} package, available at
{\tt https://svnsrv.desy.de/viewvc/marlinreco/MarlinReco/}.

The energy of {\tt SimCalorimeterHits} produced by the {\tt Mokka} simulation is the energy deposited in the detection
element (silicon or scintillator) as calculated by {\tt GEANT4}. They therefore take account of Landau fluctuations
in the energy deposits.
The role of the digitisation is to simulate the behaviour of the process which converts this energy deposit into the 
reconstructed energy of hits used in detector reconstruction: this includes effects due to the detection medium
(e.g. creation of electron-hole pairs in silicon), 
the readout system (e.g. pixelated photo-detectors (PPD - SiPM/MPPC)) used to readout scintillation light,
and the electronics which treat these signals (e.g. limited dynamic range).

This note describes a parameterised model which to model such effects. The parameters to be used should be decided by
the detector groups, by comparing to data collected during test beam campaigns.

{\tt ILDCaloDigi} also applies a threshold on the energy of hits, as well as possibilities for requirements on the timing
of energy deposits. These are not described in this note, since they were not developed by the author.

\section{General}

The various effects implemented for realistic digitisation are controlled by parameters of the {\tt ILDCaloDigi}
processor which can be set at run time via the {\tt Marlin} steering file (i.e. without recompiling).
Which type of digitisation to apply to ECAL hits is controlled by the {\tt ECAL\_apply\_realistic\_digi} parameter: a value of
0 (the default) turns off the realistic effects described in this note, and a value of 1 (2) applies the silicon--
(scintillator--) specific effects. In the case of the scintillator HCAL, the parameter {\tt HCAL\_apply\_realistic\_digi}
plays a similar role, with a value of 0 (1) turning off (on) the simulation of realistic digitisation effects.

Several digitisation parameters are specified in terms of MIP units, so the {\tt ILDCaloDigi} processor requires 
factors with which to convert the deposited energy (in GeV) to MIP units: these are passed by the {\tt CalibECALMIP} and 
{\tt CalibHCALMIP} parameters for the ECAL and HCAL, respectively.

Various detector parameters are taken from the {\tt gear} file, in particular the layer layout and number of virtual 
cells per scintillator strip. If these are not available in the {\tt gear} file, they can be specified via the
parameters {\tt ECAL\_default\_layerConfig} and {\tt StripEcal\_default\_nVirtualCells} (if values are found in the 
{\tt gear} file, they take precedence over the value of these parameters).

\section{Technology-blind effects}
\label{sec:common}

\subsection{Mis-calibrations}
The effect of imperfect energy calibrations can be simulated by the use of the parameters
{\tt ECAL\_miscalibration\_uncorrel} and {\tt ECAL\_miscalibration\_correl}, which causes hit energies to
be smeared as 
\begin{equation*}
\begin{split}
E' = & E \times \\
& \text{\tt RandGauss} ( 1, \text{\tt ECAL\_miscalibration\_uncorrel}) \times \\
& \text{\tt RandGauss} ( 1, \text{\tt ECAL\_miscalibration\_correl}) ,
\end{split}
\end{equation*}
where $\text{\tt RandGauss} (\mu, \sigma)$ represents a random number taken from a Gaussian distribution of mean $\mu$ and 
standard deviation $\sigma$. In the case of {\tt ECAL\_miscalibration\_uncorrel}, a new random number is taken for each 
calorimeter hit (simulating completely uncorrelated mis-calibrations), 
while in the case of {\tt ECAL\_miscalibration\_correl}, a single random number is used for all ECAL 
hits in a given event (simulating completely correlated mis-calibrations).

The uncorrelated miscalibrations induced by {\tt ECAL\_miscalibration\_uncorrel} of each detector cell can be chosen to be either 
the same from event to event, or newly chosen for each event. 
This is controlled by setting the parameter {\tt ECAL\_miscalibration\_uncorrel\_memorise = true} or {\tt false} respectively.
The first approach is closer to reality, however in the case of a calorimeter with many cells, can lead to large memory consumption;
in the case of typical physics events randomly spread across the whole ILD detector, the second, more memory-efficient, 
approach is almost certainly sufficient. The first approach is probably only necessary in the case of repeated injection into the same detector
region, as occurs, for example, in test beams.

\subsection{Dead detector cells}
The effect of dead detector cells can be simulated by use of the parameter {\tt ECAL\_deadCellRate}, which causes the energy
of hits to be set to zero if a random number taken from a uniform distribution in the range [0, 1] is smaller than the 
value of the {\tt ECAL\_deadCellRate} parameter.

\subsection{Dynamic range of readout electronics}
The saturation of the readout electronics can be simulated by setting the parameter {\tt ECAL\_maxDynamicRange\_MIP},
in which case the energy of the hit is limited to this value of this parameter (specified in MIP units):
\begin{equation*}
E'_{MIP} = min ( \text{\tt ECAL\_maxDynamicRange\_MIP}, E_{MIP} ).
\end{equation*}

\subsection{Noise}
Uncorrelated, random noise can be simulated by the parameter {\tt ECAL\_elec\_noise\_mips}, which
alters the hit energy by
\begin{equation*}
E'_{MIP} = E_{MIP} + \text{\tt RandGaus} ( 0, {\tt ECAL\_elec\_noise\_mips} ).
\end{equation*}

\section{Silicon ECAL hits}

A rather simple approach is followed in the case of silicon readout: 
the energy deposit in the silicon is converted into a number of electron-hole (e-h) pairs,
using the parameter {\tt energyPerEHpair} which gives the energy required to create an e-h pair (in eV).
This number of e-h pairs is then used to define the mean of a Poisson distribution, from which a random number is taken 
to get a statistically smeared number of e-h pairs. 
This approach is an over-simplification: it ignores, for example, the Fano
effect which reduces the fluctuation of e-h pairs with respect to this simple Poisson approximation.
Since these fluctuations are anyway much smaller than the Landau fluctuations in energy deposit, 
they have an almost negligible effect.

\section{Scintillator hits}

In the case of the scintillator ECAL, several effects are included: non-uniformity of response along the strip length,
and the finite number of photo-electrons and PPD pixels. 
These processes typically have much larger effects than in the case of silicon-based readout.
For the scintillator-based HCAL, the same effects, except the strip non-uniformity, are included. 
The names of the relevant parameters for the HCAL have ``{\tt ECAL\_}'' replaced by ``{\tt HCAL\_}''.

\subsection*{Non-uniformity along strip length}
In Mokka simulation, scintillator strips can be split along their length into virtual cells, 
in each of which a {\tt SimCalorimeterHit} can be produced. 
ILDCaloDigi identifies all virtual-cell hits coming from the same strip, and combines them into a single {\tt CalorimeterHit}.
Different weights can be given to the energies of different virtual cells within a strip, to simulate non-uniformity
along the strip length. A simple exponential dependence has been implemented, controlled by the parameter
{\tt ECAL\_strip\_absorbtionLength}. The energy of the final {\tt CalorimeterHit} is then given by
\begin{equation*}
E'_{MIP} = \sum_{i} E_{i} \times exp( \delta x_i / \text{\tt ECAL\_strip\_absorbtionLength} )
\end{equation*}
where the index $i$ runs over the virtual cells of a strip, and $\delta x_i$ is the distance between the 
centres of the virtual cell and the strip. This energy is then treated in the following steps:

\subsection*{Conversion of energy to MIP equivalents}
The deposited energy in the scintillator is converted to MIP units using the parameter {\tt CalibECALMIP}
\begin{equation*}
E_{MIP} = \text{\tt CalibECALMIP} \times E_{GeV} .
\end{equation*}

\subsection*{Finite number of photo-electrons}
A finite number of photo-electrons (p.e.) are produced in the PPD by energy deposited in the scintillator.
The energy deposited in the scintillator is converted to an average number of photo-electrons in the PPD:
\begin{equation*}
n_{pe}^{ave} = E_{MIP} \times \text{\tt ECAL\_PPD\_PE\_per\_MIP},
\end{equation*}
and the actual number of p.e. $n_{pe}$ is then randomly taken from a Poisson distribution with mean $n_{pe}^{ave}$.

\subsection*{Finite number of PPD pixels}
The finite number of PPD pixels introduces both a saturation effect on the average response and additional signal 
fluctuations relevant mostly at high signal levels. 
A simple model of PPD response has been implemented, which ignores cross-talk between pixels, 
after pulses, and similar effects.
The average number of fired PPD pixels ${n}_{pix}^{ave}$ for a given number of input p.e. $n_{pe}$ is modeled as
\begin{equation*}
{n}_{pix}^{ave} = \text{\tt ECAL\_PPD\_N\_Pixels} \times ( 1 - exp ( -n_{pe} / \text{\tt ECAL\_PPD\_N\_Pixels}  ) ), 
\end{equation*}
and fluctuations in the number of fired PPD pixels due to the limited number of pixels are modeled as \cite{PPD}:
\begin{equation*}
\begin{split}
n_{pix} = & {n}_{pix}^{ave} + \delta n, \text{ where} \\
\delta n = & \text{\tt RandGauss}( 0, w ) \\
       w = & \sqrt{\text{\tt ECAL\_PPD\_N\_Pixels} \times exp(-\alpha) \times ( 1 - (1+\alpha) \times \exp ( - \alpha ) ) } \\
       \alpha = & n_{pix}^{ave} / \text{\tt ECAL\_PPD\_N\_Pixels}
\end{split}
\end{equation*}

\subsection*{Variations in pixel response}
Variations in individual pixel signals (due to e.g. variations in capacitance) can be simulated by using the parameter
\text{\tt ECAL\_pixel\_spread}, which introduces extra variations in the PPD signal $n_{pe}^{sig}$
\begin{equation*}
n_{pix}^{sig} = n_{pix} \times \text{\tt RandGauss} ( 1, \text{\tt ECAL\_pixel\_spread} / \sqrt( n_{pix} ) ).
\end{equation*}

\subsection*{Electronics noise}
Electronics noise is then added to $n_{pix}^{sig}$, as described in section \ref{sec:common}.

\subsection*{Mis-calibration of total pixel number, unfolding of average PPD response}
A mis-calibration in this number of 
total PPD pixels can be introduced by the parameter {\tt ECAL\_PPD\_N\_Pixels\_uncertainty}, in which
case the assumed number of PPD pixels is defined to be:
\begin{equation*}
N_{pix} = \text{\tt ECAL\_PPD\_N\_Pixels} \times 
\text{\tt RandGauss}(1, \text{\tt ECAL\_PPD\_N\_Pixels\_uncertainty}).
\end{equation*}

This mis-calibrated number of total pixels is then used to 
unfold the PPD response by using the inverse of the saturation curve:
\begin{equation*}
n_{pe}^\text{unfold} = - N_{pix} \times log ( 1 - min ( n_{pix}^{sig}, N_{pix}-\epsilon) / N_{pix} ) ),
\end{equation*}
where $\epsilon = 1$ is a small number which ensures that the logarithm is well behaved.

\subsection*{Conversion back to energy}
The unfolded number of p.e. is then converted back into a scintillator energy deposit using the parameters
{\tt ECAL\_PPD\_PE\_per\_MIP}
and 
{\tt CalibECALMIP}.

\section{Conclusion}
Possibilities for realistic modeling of silicon-- and scintillator--based calorimeter  energy readout have been
implemented in {\tt ILDCaloDigi}. The modeling is rather simple, but should be adequate to allow 
a comparably realistic simulation of the different technologies. 
Tuning of the processor parameters
should be performed by comparisons with data collected by detector prototypes in test beams.

\begin{thebibliography}{99}
\bibitem{PPD}
A.Stoykov et al., ``On the limited amplitude resolution of multipixel Geiger-mode APDs", arXiv:0706.0746.
\end{thebibliography}

\end{document}

